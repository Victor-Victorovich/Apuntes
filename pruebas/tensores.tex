
\documentclass[11pt,a4paper,openright]{article}
\usepackage[latin1]{inputenc}
\usepackage[spanish]{babel}
\usepackage{amsmath}
\usepackage{amsfonts}
\usepackage{amssymb}
\usepackage{graphicx}
%\usepackage{draftwatermark}
%\SetWatermarkText{\textsc{Borrador}} % por defecto Draft 
%\SetWatermarkScale{5} % para que cubra toda la p�gina
%\SetWatermarkColor[rgb]{1,0,0} % por defecto gris claro
%\SetWatermarkAngle{55} % respecto a la horizontal

\newcommand{\pder}[2][ ]{\dfrac{\partial#1}{\partial#2}}
\newcommand{\der}[2][ ]{\dfrac{\mathrm{d}#1}{\mathrm{d}#2}}
\newcommand{\ppder}[2][ ]{\dfrac{\partial^2 #1}{\partial#2^2}}
\newcommand{\dder}[2][ ]{\dfrac{\mathrm{d}^2 #1}{\mathrm{d}#2^2}}
\newcommand{\vect}[1]{\boldsymbol{#1}}
\newcommand{\dv}{{\delta_v}}
\newcommand{\dif}[1]{{\mathrm{d}#1}}

\title{Capa l�mite}
\begin{document}
	%\maketitle
	
%\chapter*{Capa l�mite}	

\section*{Notaci�n tensorial y operadores diferenciales en mec�nica de fluidos}

La ecuaciones utilizadas para describir el movimiento de los fluidos utilizan  como mucho un espacio eucl�deo tridimensional. Para ello se emplea un sistema de coordenadas curvil�neas ortogonales en el espacio $E_3$ definido por la ecuaci�n:

\[\vect{s}=\vect{s}(x_1,x_2,x_3)\]
donde $\vect{s}$ es el vector posici�n y $x_i$ son los coordenadas con $i$ entre 1 y 3 en funci�n del tam�o del espacio elegido.

La variaci�n en el vector de posici�n $\vect{s}$ correspondiente a unos incrementos de las coordenadas $x_1, x_2$ y $x_3$ se puede expresar de la siguiente manera:

\[ \mathrm{d}\vect{s}=\pder[\vect{s}]{x_1}\mathrm{d}x_1 +\pder[\vect{s}]{x_2}\mathrm{d}x_2 +\pder[\vect{s}]{x_3}\mathrm{d}x_3\]

En la expresi�n anterior quedan definidos los vectores que forman la base ortogonal:

\[ \vect{a}_1=\pder[\vect{s}]{x_1}, \quad \vect{a}_2=\pder[\vect{s}]{x_2}, \quad\vect{a}_3=\pder[\vect{s}]{x_3}\]

Para transformar esta base en ortonormal, los vectores que la forman deben tener la misma direcci�n que los obtenidos, pero su m�dulo debe ser igual a la unidad. Es cambio es inmediato si se divide cada vector ortogonal por su m�dulo.

\[ \vect{e}_1=\dfrac{\partial \vect{s}/\partial x_1}{\lvert \partial \vect{s}/\partial x_1 \rvert}, \quad \vect{e}_2=\dfrac{\partial \vect{s}/\partial x_2}{\lvert \partial \vect{s}/\partial x_2 \rvert}, \quad \vect{e}_3=\dfrac{\partial \vect{s}/\partial x_3}{\lvert \partial \vect{s}/\partial x_3 \rvert}\]

Los m�dulos de los vectores ortogonales se denominan \textit{factores de escala} o \textit{par�metros m�tricos}. Empleando la letra $h$ para los factores de escala, las relaciones entre los vectores ortogonales y los vectores ortonormales que forman la base quedan:

\[ \vect{a}_1=h_1\vect{e}_1, \quad \vect{a}_2=h_2\vect{e}_2, \quad \vect{a}_3=h_3\vect{e}_3\]

De esta forma, la variaci�n del vector de posici�n queda:

\[ \dif{\vect{s}}=h_1\dif{x_1}\vect{e}_1+h_2\dif{x_2}\vect{e}_2+h_3\dif{x_3}\vect{e}_3 \]

La notaci�n anterior, cuando se tiene claro la base ortonormal utilizada, se puede simplificar escribiendo los vectores de la siguiente manera:

\[\dif{\vect{s}}=(h_1\dif{x_1},h_2\dif{x_2},h_3\dif{x_3})\]

Sea $f$ una funci�n escalar. El gradiente de esta funci�n en el sistema de coordenadas curvil�neas ortogonales expresado en vectores unitarios queda:

\[ \vect{\nabla}f=\dfrac{1}{h_1}\pder[f]{x_1}\vect{e}_1 +\dfrac{1}{h_2}\pder[f]{x_2}\vect{e}_2+\dfrac{1}{h_3}\pder[f]{x_3}\vect{e}_3\]

\end{document}